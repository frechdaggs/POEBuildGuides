% --------------------------------------------------------------
% - Fließtext-Packages
% --------------------------------------------------------------
% [Deutsches Sprachpaket]{Mehrsprachige Dokumente}
\usepackage[english]{babel}

% Umlaute im Quellcode ermöglichen.
% Dazu unbedingt die Textcodierung des Editors auch auf UTF-8 stellen
\usepackage[utf8]{inputenc}

% Font Encoding für bessere Schriftendarstellung
\usepackage[OT1]{fontenc}
\usepackage{pifont}

% Dreht bei landscape-Umgebung die Seite im .pdf
\usepackage{pdflscape}

% setzt Leerzeichen, wenn welche hingehören (vor nächstem wort, aber nicht vor "." oder ")")
\usepackage{xspace}

% ermöglicht den Zugriff auf Zeilenabstände
\usepackage{setspace}

% schönerer Blocksatz
\usepackage{microtype}

% Hurenkinder vermeiden
\usepackage[all]{nowidow}

% ermöglicht römische Zahlen
\usepackage{romannum}

% --------------------------------------------------------------
% - Fließobjekte-Packages
% --------------------------------------------------------------
% verbessert die Schnittstelle zu Fließobjekten
\usepackage{float}

% Paket zur Positionierung von Fließobjekten
\usepackage{wrapfig} 

% einbinden von Grafiken
\usepackage{graphicx}

% erzeugen von Tabellen
\usepackage{longtable}
\usepackage{array}
\usepackage{multirow}

% ermöglicht Beschriftung von Fließobjekten
\usepackage[labelfont=bf,labelsep=quad]{caption}

% Ermöglicht Darstellung von Code-Auzügen ein
\usepackage{listings}

% ermöglicht Einfärben von Tabellen
\usepackage{colortbl}


% --------------------------------------------------------------
% - Mathematik-Packages
% --------------------------------------------------------------
\usepackage[fleqn]{amsmath}
\usepackage{amsthm}
\usepackage{amssymb}
\usepackage{amsopn}
\usepackage{commath}

% Schönere Vektoren
\usepackage{esvect}

% si-Units
\usepackage[locale = DE,per-mode=fraction]{siunitx}

% --------------------------------------------------------------
% - Misc
% --------------------------------------------------------------
% Ermöglicht Blindtext als Platzhalter
\usepackage{blindtext}

% einbinden ganzer pdf-Seiten
\usepackage{pdfpages}

% verwendung von Farben
\usepackage{color}
\usepackage{xcolor}

% ermöglicht stellenweises Querformat  
\usepackage{lscape}

% ermöglicht Streichen
\usepackage{cancel}

% Zur Verwendung von bibtex
\usepackage[numbers]{natbib} 
\bibliographystyle{plainnat}

% Einstellen von Kopf- und Fußzeile
\usepackage{scrlayer-scrpage}

% --------------------------------------------------------------
% - Letztes package!!!!!!!!!!!!
% --------------------------------------------------------------
% Hyperreferenzierung	
\usepackage{url}
\usepackage{hyperref}
\usepackage[nameinlink, capitalize, ngerman]{cleveref}
\usepackage[all]{hypcap}

% Glossar
\usepackage[
nonumberlist, 	%keine Seitenzahlen anzeigen
acronyms,		%ein Formelzeichenverzeichnis erstellen
abbreviations,	%ein Abkürzungsverzeichnis erstellen
toc,          	%Einträge im Inhaltsverzeichnis
nopostdot     	%Den Punkt am Ende jeder Beschreibung deaktivieren
]{glossaries-extra}



%wichtig: Ab hier kein usepackage mehr !!!
